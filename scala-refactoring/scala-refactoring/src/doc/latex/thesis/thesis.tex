\documentclass[10pt,a4paper,oneside]{scrreprt}
\usepackage{mystyle}

\begin{document}

\begin{titlepage}
 
\begin{center}
 
\textsc{\Large University of Applied Sciences Rapperswil}
 
\textsc{\subtitlename}

\rule{0.83\textwidth}{0.4pt}

\vspace{0.2cm}

\normalsize{Spring Semester 2010}

\vspace*{\fill}
 
\Huge{\textit{\titlename}}

\vspace*{\fill}
\vspace{\baselineskip}

\normalsize{\src{scala.ifs.hsr.ch}}

\rule{0.83\textwidth}{0.4pt}
\vskip0.35cm
\begin{minipage}{0.4\textwidth}
\begin{flushleft} \large
\textsc{Author}\\
Mirko Stocker
\end{flushleft}
\end{minipage}
\begin{minipage}{0.4\textwidth}
\begin{flushright} \large
\textsc{Supervisor} \\
Prof. Peter Sommerlad
\end{flushright}
\end{minipage}
 
\end{center}
 
\end{titlepage}


\newpage

\setcounter{tocdepth}{2}

\tableofcontents

\newpage

\pagenumbering{arabic}

\chapter{Introduction}
\section{Problem Outline}
\section{Thesis Goals}

\chapter{Refactoring Library}

describes the three main modules of the refactoring library.

\section{Overview}
\section{Analysis}

explain the symbol index to resolve symbols to trees

\section{Transformation}

At the heart of every refactoring lies a \textit{transformation} that takes the current program in its abstract syntax tree form and transforms it into its refactored form. Such a transformation can be as simple as changing names -- think of the Rename refactoring -- or restructure large parts of the AST as in an Extract or Move refactoring. 

Often, a larger refactoring comprises many smaller transformations. A simple example is the Extract Method refactoring, which can be assembled from three basic transformations:

\begin{description}
 \item[Create Method] to introduce a new (empty) method.
 \item[Copy Statements] to copy the selected statements into the newly created method.
 \item[Replace Statements] to replace the original statements that have been copied to the new method with a call to it.
\end{description}

The \textit{replace} transformation itself is again a combination of two even more fundamental transformations: \textit{insert} and \textit{delete}. Once we have our Extract Method transformation, it can then again be combined with other transformations -- for example into an Extract Class refactoring. It should be clear from this that the key to a reusable refactoring library lies in the composability of its transformations. 

Conceptually, chaining simple transformations to build more powerful ones follows the Unix pipes philosophy. The design of this implementation was inspired by the Stratego program transformation tool-set (referenz) and the Kiama language processing library (referenz). Functional programming also uses the term \textit{combinator} to denote functions that can be combined and yield new functions of the same kind. An example of this are parser combinators (referenz), which are also contained in the Scala standard library.

In contrast to unix pipes that operate on their input line by line, performing transformations on trees adds an additional dimension. When transforming trees, we are also concerned with questions on how we want to traverse the tree -- i.e. top-down or bottom-up -- and to which children a transformation should be applied. The presented implementation handles all these concerns in a uniform way.

In the remainder of this section, we will develop the basics of our transformation combinators and show examples of their usage.

\subsection{Transformations}

A refactoring transformation is essentially a function that transforms a tree into an other tree. But because most transformations do not apply to all kinds of possible trees, we model a transformation as a function of type $Tree\Rightarrow Option[Tree]$, making use of Scala's \src{Option} monad to indicate inability to transform. In the actual implementation, the transformations are implemented generically as a \src{Transformation[A,~B]} that extend \src{A~$\Rightarrow$~Option[B]}:

\begin{lstlisting}
abstract class Transformation[A, B] extends (A %$\Rightarrow$% Option[B]) {
  self %$\Rightarrow$%
  def apply(in: A): Option[B]
  ...
}
\end{lstlisting}

The explicit self type annotation\footnote{In this case, think of the self type annotation as an alias for \src{this} that can be used in the body of the class.} will be used later in the implementation of the combinators. Note that all transformations are implemented generically, but to make the explanations more clear, we will assume that they are used to transform trees.

Transformations can be created from partial functions using the \src{transform} convenience function. As an example, we create a transformation that reverses the order of a class, trait, or object's member definitions and apply it to a given template instance.

\begin{lstlisting}
def transform[A, B](f: PartialFunction[A, B]) = new Transformation[A, B] {
  def apply(t: A): Option[B] = f lift t
}

val reverseTemplateMembers = transform[Tree, Tree] {
  case t: Template %$\Rightarrow$% t copy (body = t.body.reverse)
}

val result: Option[Tree] = reverseTemplateMembers(template)
\end{lstlisting}

Now that we have a way to create single transformations, we need to be able to combine them. To do this in various ways, we introduce several combinators. We use a notational shortcut to denote transformations: $A \overset{t}{_\rightarrow} [B]$ is a \src{Transformation [A, B]}.

There also exist two basic transformations, one that always succeeds, returning its input unchanged, and one that always fails, independent of its input. Depending on the context, the alias \src{id} for \src{succeed} might be a better fit and is provided as well.

\begin{lstlisting}
def succeed[A] = new Transformation[A, A] {
  def apply(a: A): Option[A] = Some(a)
}

def id[X] = success[X]

def fail[A] = new Transformation[A, A] {
  def apply(a: A): Option[A] = None
}
\end{lstlisting}

\subsection{Combinators}

There are several existing combinators already implemented in the library. On the right of each paragraph, the symbolic or alphanumeric name and type of the transformation is shown.

\paragraph{Sequence} \hfill \lstinline{*>: } $(A \overset{t}{\rightarrow} [B]) \Rightarrow (B \overset{t}{\rightarrow} [C]) \Rightarrow (A \overset{t}{\rightarrow} [C])$

\vspace{7pt} Combines two transformations so that the second one is only applied when the first one succeeded. The result of the first transformation is passed into the second one. This is implemented as the \src{andThen} method -- or alternatively with the \src{*>} operator -- on \src{Transformation}, which takes the second transformation as a by-name parameter:

\begin{lstlisting}
  def andThen[C](t: %$\Rightarrow$% Transformation[B, C]) = new Transformation[A, C] {
    def apply(a: A): Option[C] = {
      self(a) flatMap t
    }
  }
  def *>[C](t: %$\Rightarrow$% Transformation[B, C]) = andThen(t)
\end{lstlisting}


\paragraph{Choice} \hfill \lstinline{|>: } $(A \overset{t}{\rightarrow} [B]) \Rightarrow (A \overset{t}{\rightarrow} [B]) \Rightarrow (A \overset{t}{\rightarrow} [B])$

\vspace{7pt} Combines two transformations so that the second one is only applied in case the first one fails. The implementation is directly based on the underlying \src{Option} type in the \src{orElse} method on \src{Transformation} and also has an operator alias:

\begin{lstlisting}
  def orElse(t: %$\Rightarrow$% Transformation[A, B]) = new Transformation[A, B] {
    def apply(a: A): Option[B] = {
      self(a) orElse t(a)
    }
  }
  def |>(t: %$\Rightarrow$% Transformation[A, B]) = orElse(t)
\end{lstlisting}

With these two combinators, we are already able to represent conditional transformations. For example, given a transformation \src{isClass} that acts as a predicate, and two transformations \src{a} and \src{b} that represent the two possible branches the transformation can take, we can combine them into a new transformation \src{isClass *> a |> b} that executes the \src{a} transformation if the \src{isClass} transformation succeeds or \src{b} if either \src{isClass} or \src{a} fails.

Note that due to Scala's precedence rules, the \src{|>} combinator has a lower precedence than \src{*>}.

\paragraph{Predicate} \hfill \lstinline{predicate: } $(A \overset{?}{\rightarrow} Boolean) \Rightarrow (A \overset{t}{\rightarrow} [A])$

\vspace{7pt} As we have seen, transformations can be used as predicates, but usually we want to construct a predicate from a function that returns a boolean value. This can be done with the \src{predicate} function which create a transformation from a partial function.

\begin{lstlisting}
def predicate[A](f: %$\Rightarrow$% PartialFunction[A, Boolean]) = new Transformation[A, A] {
  def apply(a: A): Option[A] = if (f.isDefinedAt(a) && f(a)) Some(a) else None
}
\end{lstlisting}

\paragraph{Not} \hfill \lstinline{not: } $(A \overset{t}{\rightarrow} [A]) \Rightarrow (A \overset{t}{\rightarrow} [A])$

\vspace{7pt} A combinator that inverts a transformation. Given a transformation that succeeds, then \src{not} will fail. Should the given transformation fail, then \src{not} returns the original input unchanged. This behavior is useful for transformations that act as predicates; \src{not} can be implemented using the \src{fail} and \src{id} transformations as follows.

\begin{lstlisting}
def not[A](t: %$\Rightarrow$% Transformation[A, A]) = t *> fail |> succeed
\end{lstlisting}

Now that we have several means to specify and combine our transformations, we also need a way to apply them to a whole AST, instead of just single tree nodes. For this, there exist several traversal strategies.

\subsection{Traversal}

Applying a transformation to a single tree element is not difficult, but once we want to traverse the whole AST, we need a way to apply a transformation to all children of a tree node and to construct a new tree from the result of the transformation operation. Note that traversal strategies are also just transformations that can again be combined.

\paragraph{All} \hfill  $\forall: (A \overset{t}{\rightarrow} [B]) \Rightarrow (A \overset{t}{\rightarrow} [B])$

\vspace{7pt} Takes a transformation and creates a new one that applies the given transformation to all children, returning a single tree. Because there is no generic way to get all children and construct a new tree, we constrain the type parameter \src{A} to be convertible to $(A \Rightarrow B) \Rightarrow B$. This means that the user of the generic transformation has to pass us its children and create a new tree. When a child cannot be transformed, \src{all} immediately aborts and returns \src{None}.

\begin{lstlisting}
def all[A <%\%% (A %$\Rightarrow$% B) %$\Rightarrow$% B, B](t: %$\Rightarrow$% Transformation[A, B]) = new Transformation[A, B] {
  def apply(a: A): Option[B] = {
    Some(a(child %$\Rightarrow$% t(child) getOrElse (return None)))
  }
}
def %$\forall$% [A <%\%% (A %$\Rightarrow$% B) %$\Rightarrow$% B, B](t: %$\Rightarrow$% Transformation[A, B]) = all(t)
\end{lstlisting}

(\src{X <\% Y} is called a \textit{view bound} and demands that there exists an implicit conversion from type \src{X} to \src{Y}. This is less constrictive than \src{X <: Y}, where \src{X} has to be a subtype of \src{Y}. In our case, we can then treat \src{a} as if it were of type \src{(A $\Rightarrow$ B) $\Rightarrow$ B}. This allows us to apply the transformation to the children of \src{a}.)

\paragraph{Some} \hfill $\exists: (A \overset{t}{\rightarrow} [A]) \Rightarrow (A \overset{t}{\rightarrow} [A])$

\vspace{7pt} The \src{all} traversal only succeeds when the transformation can be applied to all children. If children that cannot be transformed should simply be kept and passed to the new tree unchanged, we can use the \src{some} transformation or its $\exists$ alias.

\begin{lstlisting}
def some[A <%\%% (A %$\Rightarrow$% A) %$\Rightarrow$% A](t: Transformation[A, A]) = %$\forall$%(t |> id[A])
def %$\exists \hspace{11pt}$% [A <%\%% (A %$\Rightarrow$% A) %$\Rightarrow$% A](t: Transformation[A, A]) = some(t)
\end{lstlisting}

Using the \src{id} transformation, we retain the original tree should the transformation not be applicable. A consequence of this is that the transformation needs to be done between the same types.

\paragraph{Once} \hfill \lstinline{once: } $(A \overset{t}{\rightarrow} [A]) \Rightarrow (A \overset{t}{\rightarrow} [A])$

\vspace{7pt} Applies a transformation and stops on the first successful application... not yet implemented.
\\\\
The next step after being able to apply a transformation to a tree or all of its children is to expand this to the AST as a whole. We can distinguish between two fundamental ways of transforming a tree: either in a top-down or bottom-up fashion.

\paragraph{Top Down} \hfill $\downarrow: (A \overset{t}{\rightarrow} [A]) \Rightarrow (A \overset{t}{\rightarrow} [A])$

\vspace{7pt} Top-down application of a transformation applies it to the parent first and then descends into its children. The consequence is that once a child gets transformed, its children are still in their original, untransformed state.

\begin{lstlisting}
  def %$\downarrow$ \hspace{19pt} %      [A <%\%% (A %$\Rightarrow$% A) %$\Rightarrow$% A](t: Transformation[A, A]) = t *> %$\forall$%(%$\downarrow$%(t))
  def topdown[A <%\%% (A %$\Rightarrow$% A) %$\Rightarrow$% A](t: Transformation[A, A]) = %$\downarrow$%(t)
\end{lstlisting}

\paragraph{Bottom Up} \hfill $\uparrow: (A \overset{t}{\rightarrow} [A]) \Rightarrow (A \overset{t}{\rightarrow} [A])$

\vspace{7pt} Bottom-up application first descends into the children of a tree and processes the parent after the children. Thus once a tree gets transformed, its children have already been transformed.

\begin{lstlisting}
  def %$\uparrow$ \hspace{27pt} %      [A <%\%% (A %$\Rightarrow$% A) %$\Rightarrow$% A](t: Transformation[A, A]) = %$\forall$%(%$\uparrow$%(t)) *> t
  def bottomup [A <%\%% (A %$\Rightarrow$% A) %$\Rightarrow$% A](t: Transformation[A, A]) = %$\uparrow$%(t)
\end{lstlisting}

Combining all these transformations with combinators and traversal strategies allows us to describe transformations in a very concise way, as we can see in the following examples.

\subsection{Examples}

As a first example, let us write a transformation that replaces all trees in the AST which do not have a range position with the \src{EmptyTree}.

\begin{lstlisting}
val hasRangePosition = predicate[Tree] {
  case t: Tree =>  t.pos.isRange
}

val emptyTree = transform[Tree, Tree] {
  case _ => EmptyTree
}

val keepTree = id[Tree]

val removeNonRangePositionTrees = %$\downarrow$%(hasRangePosition *> keepTree |> emptyTree)
\end{lstlisting}

.. hier noch mehr ..


\section{Code Generation}
explain the pretty printer and layout preserving source generator.


\chapter{Refactoring Implementations}

describe each refactoring, with examples, problems, corner cases and the actual implementation. shows the library internals from the previous chapter in action.

\section{Rename}
\section{Organize Imports}
\section{Extract Method}
\section{Extract Local}


\chapter{Integration}

how can the refactorings be integrated into a new tool?
how does the current integration look like

\chapter{Testing}

don't describe the unit tests but show how the refactoring tests work and how one could write new test cases.

\chapter{Outlook}

\section{Acknowledgments}

I would like to thank ..

\section{Future Work}

what is left to do, how can the work be extended

\appendix

\chapter{User Guide}

User guide, also in HTML on the Wiki. Describes the usage of the implemented refactoring along with their shortcomings.

\chapter{Developer How-To}

A short introduction by example, for more depth, refer to the main documentation.

\chapter{Scala Internals}

only AST and symbols, but with more detail, refer to the term project for the compiler organization and introduction.

first the general cases and then more complex and surprising examples

also build stand-alone and publish on the project page.

\chapter{Project Environment} \label{chapter:project-environment}

\chapter{Advanced Scala Features} \label{chapter:advanced-scala-features}

This documentation does not contain an introduction to the Scala language, so an understanding of the basic concepts is assumed. This appendix explains some of the more advanced features and patterns of Scala that are used during the explanations in this thesis.

\section{Path Dependent Types} \label{section:path-dependent-types}

Path dependent types are best explained with an example (taken from Programming Scala \cite{ProgrammingScala}). We have an \src{Animal} class with an abstract type member called \src{SuitableFood} which is then defined in the subclasses to a suitable type.
\begin{lstlisting}
abstract class Food

abstract class Animal {
  type SuitableFood <: Food
  def eat(food: SuitableFood)
}

class Grass extends Food
class Cow extends Animal {
  type SuitableFood = Grass
  def eat(food: Grass) {}
}

class DogFood extends Food
class Dog extends Animal {
  type SuitableFood = DogFood
  def eat(food: DogFood) {}
}
\end{lstlisting}

This now prevents us from feeding the wrong kind of food to our animals:

\needspace{11\baselineskip}
\begin{lstlisting}
scala> val bessy = new Cow
bessy: Cow = Cow@3fb01949

scala> val lassie = new Dog
lassie: Dog = Dog@46c9220

scala> lassie eat (new bessy.SuitableFood)
<console>:14: error: type mismatch;
 found   : Grass
 required: DogFood
       lassie eat (new bessy.SuitableFood)
\end{lstlisting}

In the context of the Scala compiler, all instances of \src{Tree} are dependent on the compiler -- that is, impossible to mix trees from different compiler instances.

\begin{lstlisting}
trait Trees {
  %\ldots%
  abstract class Tree extends Product {
  %\ldots%
}
\end{lstlisting}

The Scala Refactoring library follows this design, all functionality that operates on compiler dependent types is in traits that have a compiler instance as an abstract member, like for example in the \src{AbstractPrinter} or the \src{Indexes}:

\begin{lstlisting}
trait AbstractPrinter {
  val global: scala.tools.nsc.interactive.Global
  import global._
  %\ldots%
}

trait Indexes {
  val global: scala.tools.nsc.interactive.Global
  %\ldots%
}
\end{lstlisting}

The user of the refactoring library then has to provide this abstract member and all implemented traits share the same instance. In the automated tests, this is done by the \src{CompilerProvider} trait.

\section{Stackable Traits} \label{section:stackable-traits}

Stackable traits are related to the decorator design pattern, except that they do not decorate objects at run-time but traits at compile-time. Let us take a look at an example. Assume that we have a trait that allows us to log events and an implementation that logs to the standard output:

\begin{lstlisting}
trait Logging {
  def log(severity: Int, msg: String): Unit
}

class ConsoleLogger extends Logging {
  def log(severity: Int, msg: String) = {
    println("%\%%d: %\%%s" format (severity, msg))
  }
}
\end{lstlisting}

Now we want to have a logger that only logs events of a certain severity, or one that filters the messages. We could subclass \src{ConsoleLogger}, but there are potentially many concrete loggers, and we want the user of the logger to be able to combine these features as he likes. This is where stackable traits are useful. Stackable traits use the \src{abstract override} modifier to override an abstract method and are allowed to call \src{super} in the implementation, even though the super method is not implemented.

\begin{lstlisting}
trait LogOnlyErrors extends Logging {
  abstract override def log(severity: Int, msg: String) {
    if(severity >= 3)
      super.log(severity, msg)
  }
}
\end{lstlisting}

When we instantiate a \src{new ConsoleLogger with LogOnlyErrors}, the abstract override method in \src{LogOnlyErrors} overrides the \src{log} method in \src{ConsoleLogger}. We can also create more such stackable traits and combine them.

\begin{lstlisting}
trait TreatAllAsErrors extends Logging {
  abstract override def log(severity: Int, msg: String) {
    super.log(3, msg)
  }
}
\end{lstlisting}

Because of Scala's trait linearization, the order of the stackable traits is significant and allows further combinations, as shown below.

\begin{lstlisting}
scala> val logger = new ConsoleLogger with TreatAllAsErrors with LogOnlyErrors  
logger: ConsoleLogger with TreatAllAsErrors with LogOnlyErrors = $anon$1@8aee908

scala> logger.log(1, "Something insignificant happened.")

scala> logger.log(4, "A critical error, severity 4!")
3: A critical error, severity 4!

scala> val logger2 = new ConsoleLogger with LogOnlyErrors with TreatAllAsErrors
logger2: ConsoleLogger with LogOnlyErrors with TreatAllAsErrors = $anon$1@2a788315

scala> logger2.log(1, "Something insignificant happened.")
3: Something insignificant happened.

scala> logger2.log(1, "A critical error, severity 4!")
3: A critical error, severity 4!
\end{lstlisting}

\section{Implicit Conversions} \label{section:implicit-conversions}

Implicit conversions (also known as the \textit{pimp my library} pattern) can be used to (seemingly) add new methods to an existing class. Assume that we are working with currencies and have a class to represent Swiss francs:

\begin{lstlisting}
class SwissFrancs(private val amount: Int) {
  def + (other: SwissFrancs) = new SwissFrancs(amount + other.amount)
  override def toString = "CHF "+ amount
}
\end{lstlisting}

Thanks to Scala's support for methods with operator names, we can add instances of Swiss francs using \src{+}, but we still have to construct them verbosely. With an implicit conversion, we can add a \src{francs} method to \src{Int} that makes for a very readable syntax:

\begin{lstlisting}
implicit def intToSwissFrancs(i: Int) = new Object { def francs = new SwissFrancs(i) }

scala> 5.francs + 10.francs
res1:  SwissFrancs = CHF 15
\end{lstlisting}

This is also how Scala enriches Java's built-in types or why we can form tuples from any two objects using $\rightarrow$:

\begin{lstlisting}
class ArrowAssoc[A](x: A) {
  %\ldots%
  def %$\rightarrow$% [B](y: B): Tuple2[A, B] = Tuple2(x, y)
}

implicit def any2ArrowAssoc[A](x: A): ArrowAssoc[A] = new ArrowAssoc(x)

scala> "answer" %$\rightarrow$% 42
res2:  (java.lang.String, Int) = (answer,42)
\end{lstlisting}

\section{Self Type Annotation} \label{section:self-type-annotation}

Scala allows the programmer to specify an alias for \src{this} inside the current class. This is often useful to access the outer instance from an inner class where \src{this} is already bound to the inner class. 

\begin{lstlisting}
class OuterClass(val name: String) {
  outerclass =>

  class Inner {
    println("I'm the inner class of "+ outerclass.name)
  }
}
\end{lstlisting}

The self type annotation allows us to annotate this self type with additional types and are a way to describe dependencies the class or trait has. For example, if we have a class that uses some kind of service interface, we can specify the service interface with a self type annotation:

\begin{lstlisting}
trait Service {
  def callWebservice %\ldots%
}

class ServiceUser {
  self: Service =>

  callWebservice %\ldots%
}

val myService = new ServiceUser with SomeServiceImplementation
\end{lstlisting}

The user of \src{ServiceUser} than has to instantiate it with a suitable implementation of \src{Service} to make the program compile. In most cases, one could also just let \src{ServiceUser} inherit from \src{Service}, but using a self type annotation is conceptually cleared than inheritance.


\section{Package Nesting} \label{section:package-nesting}

It is a common misconception that Java supports nested packages; they are only nested in the file system, but the language itself has no notation of nested packages. In Scala on the other hand, packages can be nested. The following two listings represent different compilation units:

\needspace{4\baselineskip}
\begin{lstlisting}
package com.mycompany.project
package pd

class Student
\end{lstlisting}

\begin{lstlisting}
package com.mycompany.project
package ui

import pd.Student
\end{lstlisting}

Note how the import statement does not have to specify the fully qualified name but can simply import \src{pd.Student} because both compilation units are in the \src{com.mycompany. project} package.



\clearpage
\bib

\end{document}
